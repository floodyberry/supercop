\documentclass{article}


\usepackage{a4wide}
\usepackage[T1]{fontenc}
\usepackage[latin1]{inputenc}
\usepackage{aeguill}
\usepackage{amsmath,amssymb,amsfonts}

\newcommand{\F}{\mathbb{F}}

\title{{\tt Surf2113}: A key-exchange cryptosystem based on a Kummer
surface in characteristic 2}

\author{P. Gaudry and E. Thom�}

\date{Version 2. June 2007}

\begin{document}

\maketitle

The {\tt Surf2113} cryptosystem is a key-exchange cryptosystem based on
the Kummer surface of a genus 2 hyperelliptic curve over the finite field
$\F_{2^{113}}$. The implementation follows the formulae
described in \cite{GaudryToronto}.

\section{Parameters of the Kummer surface}

The Kummer surface that is used has the following parameters (with the
notations of \cite{GaudryToronto}):
$$ \begin{array}{rcl}
\alpha^{-1} & = & 1 \\
\beta^{-1} & = & u^9+u^3+u+1\\
\gamma^{-1} & = & u^3+u\\
\delta^{-1} & = & u^2\, ,
\end{array}$$
where $u$ is such that $u^{113}+u^9+1 = 0$. It was chosen so that the
group orders of the Jacobians of the curve and of its twist are $4$ times
a prime:
$$\begin{array}{rcl}
N & = & 4\times
26959946667150639829855262494084237597468590050436618141410242050811\\
\tilde{N} & = & 4\times
          26959946667150639759478767679955018162288000805826005897312996286579\, .
\end{array}$$

The point counting has been done using the Magma computer algebra system.

The following point $P$ on the Kummer surface has order $N/4$ and can
therefore be used as a base point in a Diffie-Hellman key exchange.
$$ P = (1, 1, 4, 908681679267597915035722095941517),$$
where the polynomials in $u$ have been written as integers by setting
$u=2$.

\noindent {\bf Security.}
We consider here the elementary security of this cryptosystem, without
taking into account the potential weaknesses of the implementation (there
are some in this version) or in the inclusion in a larger protocol
(authentication is not done, for instance). Let us then consider the best
known algorithm for solving one discrete logarithm in the $N$-order
Jacobian corresponding to this Kummer surface, namely Pollard's Rho
algorithm or distributed variants of it. This takes about the square root
of the largest prime factor of the group order: $\sqrt{N/4} \approx
2^{112}$. This is well beyond any feasible computation.

\section{Some implementation details}

\subsection{Finite field arithmetic}

Elements of the finite field are represented as a table of {\tt unsigned
long} of fixed length (2 or 4 words, depending on 32- or 64-
architecture). A polynomial basis representation is used, with a sparse
reduction polynomial. A good speed-up is obtained by using SSE-2 registers and
the corresponding 128-bit instructions. 

\subsection{Encoding of keys}

A secret key is an integer between $0$ and $2^{224}-1$. We impose it to
be a multiple of $2$ in order to avoid subgroups attacks. This secret key
is stored in big-endian form in $28$ bytes. 

The public key and the shared secret are points of the Kummer surface.
Since these are projective coordinates, we can (or we must in the case of
the share secret) make them affine by, for instance, putting the first
coordinate to $1$. There are 3 elements left to store, that all fill in
$128$ bits (actually 113), therefore it takes $48$ bytes to store a public
key or a shared secret. 

One byte could be gained for each element.

\section{Future work}

Here is our "to do" list for the next version.

\begin{itemize}
\item Key validation.

It is necessary to check that the public key of the other party is valid.
Since the curve and its twist are secure, this should be simplified.

\item Study and avoid degenerate cases.

The current version does not take into account the potential exceptional
cases that would create a non-point with all-zero coordinates. There is
some theory to do here before fixing the implementation.

\item Study the point compression.

There is some redundency in the $48$ bytes used for representing a Kummer
point. We can put all the information in $32$ bytes, but then it is
required to compute the roots of a polynomial of degree 2. We might
implement that in the next version.

\item Implement Montgomery's PRAC algorithm.

The Lucas chain we use is the classical double and add which is not
optimal. If we assume that resistance to side-channel attack is not an
issue, we can use the PRAC algorithm to save some work.


\end{itemize}

\begin{thebibliography}{10}
\bibitem{GaudryToronto}
P.~Gaudry.
\newblock Variants of the Montgomery form based on Theta functions.
\newblock Talk given at the workshop ``Computational Challenges Arising in
Algorithmic Number Theory and Cryptography'', Toronto, Novembre 2006.
\end{thebibliography}



\end{document}
