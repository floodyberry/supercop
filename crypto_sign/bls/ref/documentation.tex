\documentclass{llncs}
\usepackage{amsmath,amssymb,url}

\newcommand{\Cyc}{\mathrm{\Phi}}
\newcommand{\Frob}{\mathrm{\Phi}}
\newcommand{\twist}{\mathrm{\Psi}}
\newcommand{\F}{\mathop{\mathbb{F}}}
\newcommand{\half}[1]{{\lfloor#1/2\rfloor}}
\newcommand{\ord}{\qopname\relax{no}{ord}}
\newcommand{\sign}{\qopname\relax{no}{sign}}
\newcommand{\tr}{\qopname\relax{no}{tr}}
\newcommand{\Z}{\mathop{\mathbb{Z}}}
%\newcommand{\Cpp}{C\protect\raisebox{.18ex}{++}}
\newcommand{\Cpp}{C\kern-0.05em\texttt{+\kern-0.03em+}}

\newcommand{\C}{C}

\sloppy

\begin{document}

\pagestyle{plain}

\title{An implementation of the Boneh-Lynn-Shacham Short signature scheme}

\author{%
Michael Scott
}
\institute{
School of Computing\\%
Dublin City University\\%
Ballymun, Dublin~9, Ireland.\\%
\email{mike@computing.dcu.ie} 
}

\maketitle

\begin{abstract}

Herein we briefly document our implementation of the Boneh-Lynn-Shacham \cite{boneh-lynn-shacham} short signature 
scheme. The implementation uses the MIRACL library.

\end{abstract}

\section{Introduction}\label{sec:intro}

The Boneh-Lynn-Shacham short signature scheme, as originally described \cite{boneh-lynn-shacham}, uses the Weil 
pairing to instantiate a scheme which, for the approximate RSA-1024-bit level of security, generates a signature 
of size of about 160 bits. This is half the size of signatures based on standard methods such as ECDSA. The security of
the scheme is based on the Computational Diffie-Hellman problem over certain pairing-friendly elliptic curves, in
the random oracle model. Note that this is a classic digital signature -- it is not an identity-based signature.
It does not support message recovery.

The signature itself is the $x$ coordinate of an elliptic curve point over the prime field $\F_p$ with $p\approx 160$ bits. 
To obtain the required level of security we assume the use of a pairing-friendly curve with an embedding degree 
of $k=6$, which acts as a ``security multiplier'', and raises the index calculus strength of the scheme to ~960 bits.
Since the security of the method depends on the difficulty of the discrete logarithm problem over a sextic extension
field of this size, we are assuming that this gives approximately the same level of security as 1024-bit RSA.

We choose to implement the method using the faster Tate pairing over non-supersingular MNT curves 
\cite{miyaji-nakabayashi-takano}, \cite{scott-barreto-1}. For 
our particular chosen 
curve $p$ is actually 159 bits in length and the group size is 158 bits. In one small variation from the method as 
described in \cite{boneh-lynn-shacham}, we fix the actual signature point unambiguously by storing in the 
signature an extra bit, so that the $y$ coordinate of the point can be reconstructed (as this enables a certain 
optimization -- see below). Therefore our signature size is exactly 160-bits in length.

When using non-supersingular curves, while one parameter to the pairing may be placed on the elliptic curve over
the base field $E(\F_p)$, the other must be placed on a larger extension field. For an MNT curve the best we can
do is to place it on the quadratic twist $E'(\F_{p^3})$. Since the signature (to be short) must be on $E(\F_p)$, 
this implies 
that the public key must be over the ``bigger'' curve. So the public key will be relatively large. On the other hand 
the secret key will be just 158 bits.

We note that potentially faster methods have also been published, for example see \cite{zhang-chen-susilo-mu}, 
based on different (and arguably less compelling) security assumptions. As 
described these require the use of the modified pairing over a supersingular curve, which for an embedding degree 
of 6 requires a fast implementation of characteristic 3 arithmetic, which we do not have to hand.

Note that the timings given in the original paper are not representative of the potential of the method. We are 
aware of implementations (using a characteristic 3 supersingular curve) which are much faster than those reported in 
the paper.

\section{The implementation}

The fixed domain parameters consist of the pairing friendly MNT curve with an embedding degree of 6, and a fixed
point $P$ on the twisted curve $E'(\F_{p^3})$. The private signing key is a 20 byte randomly generated number $s$.
The public key is the point $R=sP$. $R$ will be 120 bytes in size, although using point compression this could be
reduced to 60 bytes.

To sign a message it must first be hashed to a point $Q$ on the curve $E(\F_p)$. Basically the message is hashed
to an $x$ coordinate, and if the RHS of the elliptic curve equation is a quadratic residue (tested by calculating 
a jacobi symbol), then a modular square root determines the $y$ coordinate. If the RHS of the elliptic curve equation
is not a quadratic residue, the hash value is incremented until it becomes one. The total cost will thus be one modular 
square root and one or more jacobi symbol calculations. The signature is then the point $S=xQ$, compressed using 
standard methods to its $x$ coordinate and a single bit, to a size of 20 bytes.

To verify the message we reconstruct $S$ from its compressed form, hash the message again to the point $Q$
and test that $e(Q,R).e(-S,P) = 1$. This should be the case as 

$$ e(Q,R).e(-S,P) = e(Q,xP).e(-xQ,P) = e(Q,xP)/e(Q,xP) = 1 $$

Doing it this way the tricks for calculating a product of pairings \cite{scott} can be used, with 
some savings.


\bibliographystyle{plain}
\bibliography{bls}


\end{document}
